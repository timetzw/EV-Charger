
\subsection{Background}
Electric Vehicles (EVs) have become increasingly popular in recent years, as they offer numerous environmental and economic benefits. With the growing adoption of EVs, the demand for efficient and accessible charging infrastructure has risen significantly. An important aspect of developing this infrastructure is determining the optimal allocation of EV chargers to maximize utility for users. Factors such as traffic flows and pricing strategies play a crucial role in the decision-making process for allocating charging stations.

The objective of this paper is to investigate the impact of different traffic flows and pricing strategies on the optimal allocation of EV chargers. Specifically, we focus on two distinct traffic flow scenarios: normal days and game days. By analyzing the effects of these scenarios on charger allocation, we aim to provide valuable insights that can inform policymakers, EV charging infrastructure providers, and other stakeholders in their decision-making process.

To achieve our objectives, we will employ a simulation model that takes into account various factors, including traffic flows, pricing strategies, and charger availability. The model will use a utility function to evaluate different allocation plans, ultimately identifying the best allocation strategy under each scenario.

Through this research, we hope to contribute to the existing body of knowledge on EV charger allocation and offer practical recommendations for optimizing charging infrastructure. By considering the influence of traffic flows and pricing strategies on charger allocation, we aim to promote the efficient and sustainable growth of the EV market, ultimately benefiting both users and the environment.

\subsection{Literature Review}

In this section, we will review the existing literature on EV charger allocation, traffic flows, and pricing strategies to provide context for our study and identify areas where our research can make a valuable contribution.

EV charger allocation: Numerous studies have explored various aspects of EV charger allocation, such as placement strategies, charger types, and the role of geographical factors. These studies have helped to establish a foundation for understanding the complexities of allocating EV chargers in a way that maximizes utility for users.

\textbf{Details to be added.}
% Study 1 (arXiv link) - Discuss the main findings of this study related to charger allocation.
% Study 2 (arXiv link) - Explain the contributions of this study to the field of EV charger allocation.
% Traffic flows: The impact of traffic flows on EV charger allocation has been an important area of investigation. Researchers have examined how different traffic patterns, such as daily fluctuations, seasonal variations, and special events, can influence the demand for EV charging and the optimal allocation of charging stations.

% Study 3 (arXiv link) - Summarize the main findings of this study related to traffic flow impacts on EV charger allocation.
% Study 4 (arXiv link) - Describe the insights provided by this study on the relationship between traffic flows and EV charger allocation.

% Pricing strategies: Pricing strategies play a significant role in determining the usage patterns of EV charging stations and, consequently, the optimal allocation of chargers. Several studies have focused on the effects of different pricing strategies, such as time-of-use pricing, dynamic pricing, and flat rate pricing, on the demand for EV charging and charger allocation decisions.

% - *Study 5* (arXiv link) - Discuss the main findings of this study related to the impact of pricing strategies on EV charger allocation.
% - *Study 6* (arXiv link) - Explain the contributions of this study to the understanding of pricing strategies and their effects on charger allocation.

% In summary, the existing literature has provided valuable insights into various aspects of EV charger allocation, including the influence of traffic flows and pricing strategies. However, there remain gaps in the literature, particularly in terms of the combined effects of these factors on charger allocation under specific scenarios such as normal days and game days. Our study aims to address these gaps by conducting a detailed simulation-based analysis of the impact of different traffic flows and pricing strategies on the optimal allocation of EV chargers.


% Electric vehicle (EV) charging infrastructure is a rapidly evolving field, with various researchers and organizations working to improve the technology and make it more widely available. Current EV chargers can be broadly categorized into three types: Level 1, Level 2, and DC fast charging. Level 1 chargers are the most basic and slowest, typically using a standard 120-volt household outlet to charge the EV. Level 2 chargers use a 240-volt outlet, similar to that used for an electric dryer or range, and can charge an EV much more quickly. DC fast chargers, also known as Level 3 chargers, use direct current (DC) power to charge an EV in a matter of minutes rather than hours.

% Researchers have made significant contributions to the development of EV charging infrastructure, including the development of more efficient and cost-effective charging equipment, the expansion of charging networks, and the integration of renewable energy sources into the charging process. For example, a team from the University of California, Davis developed a model to optimize the placement of EV charging stations, taking into account factors such as the number of potential users, the availability of renewable energy sources, and the cost of the charging equipment. Another research group at the Technical University of Denmark developed a system that uses solar panels and energy storage to power EV charging stations, reducing the environmental impact of the charging process.

% In addition to the development of EV charging equipment, researchers have also focused on the problem of EV charger distribution or allocation design. This involves determining the optimal locations for EV charging stations, taking into account factors such as the number and distribution of potential EV users, the availability of renewable energy sources, and the cost of the charging equipment.

% One of the main challenges in this area is to ensure that charging infrastructure is accessible to as many EV users as possible, while also minimizing the costs associated with building and maintaining the charging stations. Researchers have used various methods to address this challenge, including mathematical modeling, network optimization, and machine learning techniques.

% For example, a team from the Technical University of Munich used a mathematical model to optimize the placement of EV charging stations in urban areas, taking into account factors such as the distribution of EV users and the availability of renewable energy sources. Another research group at the University of California, Berkeley used a network optimization approach to determine the optimal locations for EV charging stations on a regional scale, considering the potential demand from EV users and the cost of building and operating the charging stations.

% Additionally, researchers have employed machine learning techniques to predict the demand for EV charging stations and optimize their distribution. For instance, a team from the University of California, San Diego used a machine learning algorithm to predict the demand for EV charging stations in a given region and determine the optimal locations for the charging stations based on the predicted demand.

% However, with respect to the specific parking and charging details, previous researches have not simulated the details. In this paper, we expect to use Madison city as an example to do detailed simulation of the charging behavior, and would like to see how these behaviors going to change the final optimal allocation design.