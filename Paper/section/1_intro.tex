
\subsection{Background}

With the global climate crisis mounting and the constant pressure on mankind to transform energy consumption habits, a significant pivot is occurring from internal combustion engine (ICE) vehicles to electric vehicles (EVs). EVs, powered by electricity, have emerged as a sustainable and greener alternative, significantly reducing greenhouse gas emissions and decreasing our reliance on fossil fuels. This substantial shift towards electrification of transport, propelled by both policy measures and technological advancements, presents new challenges and opportunities that need to be adequately addressed to maximize benefits and ensure the success of this transition.

Recent years have witnessed a surge in EV popularity, primarily driven by supportive government policies, growing environmental awareness, and advancements in battery technology. However, the ubiquity of EVs is not without its challenges. The increasing demand and adoption of EVs have stressed the existing charging infrastructure, rendering it incapable of supporting the present growth rate. As EVs continue to permeate the market at an accelerating pace, the limitations and inefficiencies of the existing charging infrastructure have become more prominent.

This highlights the necessity of effective charger allocation, which is paramount for the continued growth and acceptance of electric vehicles. Efficient charger allocation has the potential to enhance user experience by reducing waiting times at charging stations, minimizing range anxiety, and further facilitating the adoption of EVs. Moreover, it has significant implications for the grid, as it can help balance loads, prevent blackouts, and encourage efficient energy use, while also paving the way for innovative demand response strategies.

However, optimizing charger allocation is a complex task that must take into account numerous variables, such as the spatial and temporal patterns of EV charging demand, the capacity of the electrical grid, and the various types of charging equipment. 

Given this complexity and the crucial role of efficient charger allocation in the widespread adoption of EVs, there is a pressing need for a comprehensive study on this subject. 

\subsection{Literature Review}
%%TODO
In this section, we will review the existing literature on EV charger allocation, traffic flows, and pricing strategies to provide context for our study and identify areas where our research can make a valuable contribution.

EV charger allocation: Numerous studies have explored various aspects of EV charger allocation, such as placement strategies, charger types, and the role of geographical factors. These studies have helped to establish a foundation for understanding the complexities of allocating EV chargers in a way that maximizes utility for users.

\textbf{Details to be added.}

\subsection{Objectives}
In this paper, we will investigate the impact of different traffic flows and pricing strategies on the optimal allocation of EV chargers. Specifically, we will focus on two distinct traffic flow scenarios: normal days and game days. By analyzing the effects of these scenarios on charger allocation, we aim to provide valuable insights that can inform policymakers, EV charging infrastructure providers, and other stakeholders in their decision-making process.
